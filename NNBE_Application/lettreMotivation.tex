\documentclass[a4paper,oneside,12pt]{article}

\usepackage[french]{babel}
\frenchbsetup{StandardLists=true}
\usepackage{enumitem}
\usepackage[T1]{fontenc} 
\usepackage[utf8]{inputenc}
\usepackage{graphicx}
\usepackage{amsfonts}
\usepackage{amsmath}
\usepackage{amsthm}
\usepackage{empheq}
\usepackage{eurosym}
\usepackage{subfigure}
\usepackage{multicol}
\usepackage{color}
\usepackage{ulem}
\PassOptionsToPackage{hyphens}{url}\usepackage{hyperref}
\usepackage{geometry}
\usepackage{fancybox}
\usepackage[final]{pdfpages} 
\geometry{hmargin=2cm, vmargin=2.5cm}
%\setlength{\parindent}{0cm}

\hypersetup{colorlinks, linkcolor=blue}

\begin{document}

\thispagestyle{empty}




\begin{flushleft}
GREINER Nathan

163 rue du Chevaleret \hspace{6.5cm} A Paris, le 6 Septembre 2013,

75013 Paris

\underline{nathan.greiner@polytechnique.edu}

07.81.80.22.44
\end{flushleft}

\vspace{2cm}

\begin{flushright}
A l'attention des personnes concernées,
\end{flushright}

\vspace{1cm}

Madame, Monsieur, \\

Elève de l'Ecole Polytechnique, je viens d'achever ma troisième et dernière année dans cette institution.

Extrêmement intéressé par les métiers et les problématiques liés à l'énergie, aussi bien celle d'origine nucléaire que renouvelable (éolienne, solaire, \dots), j'ai choisi de suivre au cours de cette troisième année un parcours intitulé \og Energies du XXI$^{\textrm{ème}}$ siècle \fg. J'y ai notamment suivi un nombre conséquent de cours portant sur l'énergie nucléaire :

\begin{itemize}
\item{\og Physique des réacteurs nucléaires \fg, par Sylvain David,}
\item{\og Technologie des réacteurs nucléaires et cycle du combustible \fg, par Frank Carré,}
\item{\og Fusion thermonucléaire \fg, par Patrick Mora.}
\end{itemize}
J'ai pris beaucoup de plaisir à étudier ces sujets, et mon intérêt au cours de cette année scolaire n'a fait que grandir.

Malgré la mauvaise presse dont elle fait l'objet en ce début de siècle, en cause les accidents comme celui, déjà trop cité, de Fukushima, cette source d'énergie me semble résolument être une solution d'avenir pour assurer de façon pérenne et faiblement émettrice de gaz à effet de serre la production d'électricité dans le monde entier. De nombreux pays, comme la Chine ou l'Inde, ont d'ailleurs déjà fait le choix de persévérer dans leur programme de développement nucléaire en dépit de l'accident japonais.

Pour ces raisons, je m'apprête à entamer une maîtrise de recherche en génie nucléaire à Montréal, Canada, à compter du mois de Janvier 2014. 

D'ici là, je désire ardemment effectuer un stage, par exemple au sein du CNEN, pour une première expérience dans ce monde qui m'est pour l'instant inconnu. J'aspire à ce qu'une ou plusieurs tâches précises me soient confiées au sein d'un groupe de travail, et m'efforcerai alors de les accomplir au mieux. \\

Je vous prie de croire, Madame, Monsieur, en mes sentiments respectueux, \\

\vspace{5mm}

\hspace{10cm} Nathan Greiner





\end{document}