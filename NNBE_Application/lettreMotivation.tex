\documentclass[a4paper,oneside,12pt]{article}

\usepackage[english]{babel}
\usepackage{enumitem}
\usepackage[T1]{fontenc} 
\usepackage[utf8]{inputenc}
\usepackage{graphicx}
\usepackage{amsfonts}
\usepackage{amsmath}
\usepackage{amsthm}
\usepackage{empheq}
\usepackage{eurosym}
\usepackage{subfigure}
\usepackage{multicol}
\usepackage{color}
\usepackage{ulem}
\usepackage[super]{nth}
\PassOptionsToPackage{hyphens}{url}\usepackage{hyperref}
\usepackage{geometry}
\usepackage{fancybox}
\usepackage[final]{pdfpages} 
\geometry{hmargin=2cm, vmargin=2.5cm}
%\setlength{\parindent}{0cm}

\hypersetup{colorlinks, linkcolor=blue}

\begin{document}

\thispagestyle{empty}




\begin{flushleft}
GREINER Nathan

112 route de Vallaire \hspace{6.5cm} Lausanne, June \nth{1}, 2016

1024 Ecublens, SUISSE

nathan.greiner@epfl.ch

+41 76 531 02 94
\end{flushleft}

\vspace{2cm}

\begin{flushright}
\textbf{\uline{Object}}: Motivation letter for the 2016th NNBE summer school.
\end{flushright}

\vspace{1cm}

Have you ever paid a sharp attention to this tiny aggregate of undifferentiated cells that start multiplying through mitosis, specializing and organizing into highly functional and reliable units - the organs - and eventually, after billions of years of reproduction and evolution, produce such a complex organism as the human being ?

Conductor among the organs, the brain, together with the peripheral nervous system, allows us to produce limb movements, experience emotions or control our heart rate. \\

However, these outstanding abilities can be lost. After a spinal cord injury, the connections between the brain and the spinal circuits below the lesion may be interrupted, which can lead to paraplegia or tetraplegia and to severe impairments in people's everyday life.

But research in neurophysiology and development of neuroprosthetics show promising strategies and have already demonstrated a strong potential to alleviate the consequences of spinal cord injuries. \\

I am working as a research assistant in the Courtine-Lab at the École Polytechnique Fédérale de Lausanne, Switzerland, where I dedicate my time to the development of comprehensive computational models of epidural electrical stimulation, a technique for the recovery of sensorimotor functions in paralyzed patients.

I intend to follow a PhD program in this institution and to push forward my work and contribute to the findings for curing paraplegia and tetraplegia throughout the world. \\

That is why I hope I will be able to attend this summer’s NNBE, during which I would benefit from lectures and exchanges with great experts of the field. \\

\vspace{2mm}

Yours sincerely, 

\begin{flushright}
Nathan GREINER
\end{flushright}





\end{document}