\documentclass[a4paper,oneside,12pt]{article}

\usepackage[english]{babel}
\usepackage{enumitem}
\usepackage[T1]{fontenc} 
\usepackage[utf8]{inputenc}
\usepackage{graphicx}
\usepackage{amsfonts}
\usepackage{amsmath}
\usepackage{amsthm}
\usepackage{empheq}
\usepackage{eurosym}
\usepackage{subfigure}
\usepackage{multicol}
\usepackage{color}
\usepackage{ulem}
\usepackage[super]{nth}
\PassOptionsToPackage{hyphens}{url}\usepackage{hyperref}
\usepackage{geometry}
\usepackage{fancybox}
\usepackage[final]{pdfpages} 
\geometry{hmargin=2cm, vmargin=2.5cm}
%\setlength{\parindent}{0cm}

\hypersetup{colorlinks, linkcolor=blue}

\begin{document}

\thispagestyle{empty}




\begin{flushleft}
GREINER Nathan

112 route de Vallaire \hspace{6.5cm} Lausanne, June \nth{1}, 2016

1024 Ecublens, SUISSE

nathan.greiner@epfl.ch

+41 76 531 02 94
\end{flushleft}

\vspace{1.5cm}

\begin{flushright}
\textbf{\uline{Question}}:  What is your vision about what will be the next breakthrough in the field on Neurophysiology/Neuroprosthetics in the next 10 years and how it will transform the field ?
\end{flushright}

\vspace{0.8cm}

It is possible - though difficult - to draw a picture of what could be the state-of-the-art of neurophysiology and neuroprosthetics 10 years forward from now. But while talking about breakthroughs in our discipline, we must interrogate which kind of spectacular outcomes we are expecting and on which level : will it be new fundamental understandings of the organization and mechanisms of specific subsets of neurons driving given motor actions ? Developments of new electrical stimulation and recording paradigms further enhancing the abilities of impaired patients ? Or large-scale spreading of economically affordable and clinically approved devices for the daily use of people suffering from neurologic disorders ?

However, simply trying to predict the future - just as being optimistic or pessimistic about the result of a football game - is a spectator attitude. On the contrary, when you are a player on the field, you get prepared for the challenges to overcome and struggle to shape the future which you want to build. \\

In my opinion, research in therapeutic strategies for alleviating the consequences of tetraplegia following a spinal cord injury, not only is crucial for the patients and for unloading society with the cost that it represents, but is also offering a wide space for achieving significant progress. 
As was recently reported in a study on locomotion (ref), the mechanistic understanding of the interaction of epidural electric stimulation and the lumbar spinal circuits of the rat allowed the development of a closed-loop system enabling a subject-specific correction of gait and balance during locomotion.

Comparable findings and paradigms developments for the upper-limb are yet to come. The unraveling of the cervical spinal cord mechanisms to control arm movements could lead the way to clinical applications positively and critically improving the patients’ everyday lives.
I think that the development of new computational models and frameworks – to which I will dedicate the four years to come as a PhD student - will contribute to this multilevel - fundamental and applied - breakthrough in the field. \\

\end{document}